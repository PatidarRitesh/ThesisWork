\documentclass[letterpaper, 10 pt, conference]{ieeeconf}  % Comment this line out if you need a4paper

%\documentclass[a4paper, 10pt, conference]{ieeeconf}      % Use this line for a4 paper

\IEEEoverridecommandlockouts                              % This command is only needed if 
                                                          % you want to use the \thanks command

\overrideIEEEmargins      
\usepackage{amsmath} % assumes amsmath package installed
\usepackage{amssymb}  % assumes amsmath package installed
\usepackage{dsfont}
\usepackage{graphicx}

\usepackage{psfrag,graphicx,epsfig}
\usepackage{epstopdf}
\usepackage{xspace}
\usepackage{subfig}
\usepackage{float}
\usepackage{placeins}
\usepackage{multirow}
\usepackage{pgf,tikz}
\usepackage{nowidow}
\usepackage{lineno}
\usepackage{xcolor}
%\usepackage{arydshln} 
\newcommand{\fig}[1]{Fig.~\ref{#1}}
\newcommand{\tab}[1]{Table~\ref{#1}}
\newcommand{\eq}[1]{(\ref{#1})}
\DeclareMathOperator*{\argmin}{arg\,min}
\DeclareMathOperator*{\argmax}{arg\,max}
\newcommand{\alg}[1]{Alg.~\ref{#1}}
\captionsetup{font=footnotesize}
\usepackage{siunitx}
\usepackage{color}
\usepackage{flushend}
\usepackage{lineno}
\usepackage{algorithm}
\usepackage[noend]{algorithmic}% Needed to meet printer requirements.
\allowdisplaybreaks
\newcommand{\djnote}[1]{\textcolor{blue}{\textbf{DJ: #1}}}
%In case you encounter the following error:
%Error 1010 The PDF file may be corrupt (unable to open PDF file) OR
%Error 1000 An error occurred while parsing a contents stream. Unable to analyze the PDF file.
%This is a known problem with pdfLaTeX conversion filter. The file cannot be opened with acrobat reader
%Please use one of the alternatives below to circumvent this error by uncommenting one or the other
%\pdfobjcompresslevel=0
%\pdfminorversion=4

% See the \addtolength command later in the file to balance the column lengths
% on the last page of the document

% The following packages can be found on http:\\www.ctan.org
%\usepackage{graphics} % for pdf, bitmapped graphics files
%\usepackage{epsfig} % for postscript graphics files
%\usepackage{mathptmx} % assumes new font selection scheme installed
%\usepackage{times} % assumes new font selection scheme installed
%\usepackage{amsmath} % assumes amsmath package installed
%\usepackage{amssymb}  % assumes amsmath package installed

\title{\LARGE \bf
Covariance Steering for  Uncertain Contact-rich Systems}
% Robust Pivoting using Frictional Stability

\author{Yuki Shirai$^{\dagger}$, Devesh K. Jha$^{\ddagger}$, and Arvind U. Raghunathan$^{\ddagger}$%   
\thanks{$^{\dagger}$ Yuki Shirai is with the Department of Mechanical and Aerospace Engineering, University of California, Los Angeles, CA, USA 90095 {\tt\small yukishirai4869@g.ucla.edu}}%
\thanks{$^{\ddagger}$Devesh K. Jha and Arvind U. Raghunathan are with Mitsubishi Electric Research Laboratories (MERL), Cambridge, MA, USA 02139 {\tt\small \{jha,raghunathan\}@merl.com}}}%


\begin{document}



\maketitle
% \thispagestyle{empty}
% \pagestyle{empty}
% \twocolumn[{%
%     \renewcommand\twocolumn[1][]{#1}%
%     \maketitle
%     \begin{center}
%         \centering
%         \includegraphics[width=.8\textwidth]{Figures/pivot_fig1.eps}
%         \captionof{figure}{
%             We consider the problem of reorienting parts for assembly using pivoting manipulation primitive. Such reorientation could possibly be required when the parts being assembled are too big to grasp in the initial pose (such as the gears) or the parts to be inserted during assembly are not in the desired pose (such as the pegs). We present analysis showing that friction can provide stability to uncertainty in mass and COM location of the object being manipulated. We present a bilevel trajectory optimization method that can maximize this frictional stability under some simplifying assumptions. The figure shows some instances during implementation of our computed controller to reorient the gear and peg.  %Since the SL generates feasible waypoints for arbitrary environments and RL learns to follow the arbitrary paths, the agent can generalize to novel environments as well as achieving good sample efficiency.
%             % \konote{Rephrase}
%             % \aknote{Is it possible to visualize what is our key idea of generalization?}
%         }
%         \label{fig:pivoting_abstractfig}
%     \end{center}%
%     }]
%     \footnotetext[1]{$^{\dagger}$ Yuki Shirai is with the Department of Mechanical and Aerospace Engineering, University of California, Los Angeles, CA, USA 90095 {\tt\small yukishirai4869@g.ucla.edu}}%
%   \footnotetext[2]{$^{\ddagger}$Devesh K. Jha, Arvind U. Raghunathan and Diego Romeres are with Mitsubishi Electric Research Laboratories (MERL), Cambridge, MA, USA 02139 {\tt\small \{jha,raghunathan,romeres\}@merl.com}}%
  %\footnotetext[3]{Devesh K. Jha is with Mitsubishi Electric Research Labs, Cambridge, MA, USA.
   %{\tt\small jha@merl.com}}


%%%%%%%%%%%%%%%%%%%%%%%%%%%%%%%%%%%%%%%%%%%%%%%%%%%%%%%%%%%%%%%%%%%%%%%%%%%%%%%%

\begin{abstract}
 Planning and control for uncertain contact systems is challenging as it is not clear how to propagate uncertainty for planning. Contact-rich tasks can be modeled efficiently using complementarity constraints among other techniques. In this paper, we present a stochastic optimization technique with chance constraints for systems with stochastic complementarity constraints. We use a particle filter-based approach to propagate moments for stochastic complementarity system. To circumvent the issues of open-loop chance constrained planning, we propose a contact-aware controller for covariance steering of the complementarity system. Our optimization problem is formulated as Non-Linear Programming (NLP) using bilevel optimization. We present an important-particle algorithm for numerical efficiency for the underlying control problem. We verify that our contact-aware closed-loop controller is able to steer the covariance of the states under stochastic contact-rich tasks.
%  design trajectories with chance constraints for contact-rich tasks.
%  using the proposed closed-loop controller, we can achieve 
%  better performance when compared to the stochastic open-loop controller. 

%The proposed method is demonstrated for optimization in several contact-rich systems as well as a manipulation example with a physical 6 DoF manipulator arm. 

% Planning with uncertain complementarity systems is challenging as it is not clear how to propagate uncertainty 
% In this paper, we present a method for stochastic optimization in contact-rich systems with uncertainty. 

% Stochastic model predictive control (MPC) is a very powerful approach to perform control of uncertain systems. 

% However, stochastic MPC for contact-rich systems is challenging for nonlinear systems as exact uncertainty propagation is difficult for nonlinear systems. Furthermore, the solutions to stochastic MPC problems could be very conservative. In this paper, we present a technique for performing stochastic optimization in nonlinear systems using covariance steering. 
%     Our proposed method makes use of an exact method for uncertainty propagation and a feedback structure of the controller to allow better control of the uncertain nonlinear system. The proposed approach could also be useful for model-based reinforcement learning problems where the proposed controller design allows faster learning rates during learning. The proposed uncertainty aware optimization is demonstrated on several different systems as well as for a pendulum system during MBRL.
\end{abstract}
\input{introduction}
\input{related_works}
\input{problem_statement}
\input{cov_control}
\input{results}
\input{discussion}

%%%%%%%%%%%%%%%%%%%%%%%%%%%%%%%%%%%%%%%%%%%%%%%%%%%%%%%%%%%%%%%%%%%%%%%%%%%%%%%%

\bibliographystyle{IEEEtran}
\bibliography{main}

\end{document}
